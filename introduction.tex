\section{Introduction}
\hrule height 1pt
\vspace*{5pt}
The problem of projecting Earth's spherical surface onto a two-dimensional 
plane stands as one of the most persistent intellectual challenges in human 
history (Snyder, 1987). Since antiquity, this fundamental limitation of 
cartography has compelled scholars to develop increasingly complex 
mathematical solutions while confronting unavoidable geometric compromises. 
Far from being merely technical instruments for navigation, maps are powerful 
cultural artifacts that reveal how societies conceptualise space, territory 
and their prominence in the world (Harley, 1989). 

One of the earliest systematic treatments of cartography was established in 
the 2nd century CE by Claudius Ptolemy, whose Geographia introduced the 
revolutionary concept of a coordinate grid system (Claudius et al., 2002).  
This pioneering work not only shaped medieval and Renaissance cartographic 
development for centuries but also demonstrated the inherent mathematical 
nature embedded in this problem (Thrower, 2008). 

During the Islamic Golden Age, (8th to 13th century), scholars such as 
al-Khwārizmī and al-Idrīsī advanced mathematical geography by synthesising 
Ptolemaic theory with new empirical data from Arab navigators and astronomers 
(Edson et al., 2011). Parallel developments occurred in East Asia, where 
Chinese cartographers created sophisticated grid-based maps for administrative 
and military purposes, demonstrating the cross-cultural importance of spatial 
representation (Yee, 1994). 

A major breakthrough in projections came during the Age of Exploration (15th 
to 17th century), as European empires ventured across the vast oceans and the 
demand for more precise navigation tools intensified (Brotton, 2014). This 
period witnessed the development of several landmark projections, most notably 
Gerardus Mercator's 1569 cylindrical projection. While its conformal 
properties proved invaluable for marine navigation, its severe areal 
distortion at high latitudes later became a subject of considerable criticism 
(Battersby et al., 2014). Contemporary cartographers including Johannes 
Stabius, Johannes Werner, and Oronce Finé developed alternative projections 
such as azimuthal and cordiform maps, reflecting the growing interest and 
recognition of the trade-offs involved in flattening a sphere (Bugayevskiy \& Snyder, 2013). 

While cartographers grappled with practical representations of Earth, 
mathematicians were developing the theoretical foundations that would later 
revolutionise map projections. The origins of differential geometry can be 
traced to the 17th and 18th centuries, when Christiaan Huygens and Leonhard 
Euler began systematically studying the curvature of surfaces (Struik, 1988). 
Their work on osculating circles and developable surfaces laid the groundwork 
for understanding how three-dimensional forms could be mathematically 
described.

The two field of mathematics and cartography collided most dramatically in the 
20th century, when mathematicians began applying differential geometry to 
optimise map projections (Snyder, 1987). Today, modern computational maps 
continue to build upon this mathematical foundation, powering navigation 
systems worldwide. 

Against this historical backdrop, this Extended Essay aims to investigate the 
Research Question:

\textbf{How do we quantify geometric distortions of map projections using the tools from 
differential geometry?}

I will explore how tools such as the First Fundamental Form, Gaussian 
curvature, and Tissot's Indicatrix allow us to analyse and visualise 
distortions in length, area, and angle on maps. By modelling Earth as a 
perfect sphere, I will examine three widely used projections - the Mercator, 
Mollweide, and Lambert Conformal Conic - and use my mathematical framework to 
calculate the geometric distortions each introduces.
