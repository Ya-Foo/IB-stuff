\section{\textsc{Evaluation}}
\hrule height 0.5pt
\vspace*{2.5pt}

While the outcome of this investigation was rather successful, there were notable assumptions and methodological limitations that could 
have been better addressed and improved.

One of the most significant setbacks was the initial decision to generate terrain maps at a resolution $1024\times1042$ pixels. While the intention 
was to match the granularity of real-world terrain, this approach quickly proved computationally efficient, especially for the Moran's I calculations 
which took at least 2 hours to finish for one map. To ensure feasibility within the timeline, I had to reduce the resolution to $256\times256$. Although 
this allowed for smoother experimentation, it introduced the trade off that the lower resolution may not have captured finer-scale topographic features, 
potentially limiting the fairness of direct comparisons with real-world datasets. Even when I tried to resample real-world datasets to match the $256\times256$
resolution of the noise maps, it smoothed and altered the fine-scale features in the original terrain data, which downgraded the accuracy of spatial and 
spectral comparisons. In future investigations, a multiscale analysis could be adopted to examine noise performance at multiple resolutions.  

Moreover, several assumptions were made without rigorous justification. For instance, while constructing the smoothstep functions used in interpolation, I assumed 
that the order must be odd to satisfy a symmetry condition. Although this assumption aligns with some sources, I was unable to produce a formal mathematical proof. 
This is an area where greater rigor could be applied in future iterations of this investigation. 

A similar limitation concern the preference of Moran's I over Geary's C in evaluating spatial autocorrelation. While I justified this based on the focus of my 
analysis on global similarity patterns, I did not conduct a comparative analysis of the two metrics, nor did I examine cases where Geary's C might yield extra 
helpful insights. Including such a comparison could have added depth to my conclusions and strengthened the reliability of the findings. 

Another methodological refinement that could enhance the evaluation of realism is the classification of noise colour. Terrain surfaces, like many natural phenomena, 
often exhibit red noise characteristics, where lower spatial frequencies dominate. Incorporating spectral slope analysis to determine whether the noise resembles 
red, pink, white, or blue noise would offer an additional dimension of realism assessment. Since human perception tends to associate “naturalness” with red noise 
(Haroz, 2006), this framework could help formalize qualitative judgments of visual realism.

I had also utilized only one set of parameters to perform the comparison, which discards numerous other possibilities that could have enhanced the synthetic noise 
similarities to real terrain. This was done because otherwise it would go beyond the scope of this IA. However, in the future, I would investigate with different 
parametric settings; although I cannot include all in the paper, at least I could compile into a concise table and put in the Appendix. 

Finally, a limitation inherent in my methodology was the reliance on a fixed seed for initial comparisons. Larger sample size could have been employed to more 
robustly quantify the variance and confidence in the conclusion. 
