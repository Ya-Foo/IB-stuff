\section{\textsc{Conclusion}}
\hrule height 0.5pt
\vspace*{2.5pt}

In conclusion, I successfully achieved my aim of the investigation and demonstrated the superiority of the Perlin noise function in 
simulating realistic terrain features. This was accomplished through a comprehensive evaluation based on both statistical measures, such 
as the Moran's I spatial autocorrelation, and spectral characteristics.

In terms of basic elevation statistics, Perlin noise exhibits a higher mean elevation and greater variance than Value noise. These metrics 
align more closely with natural landscapes, suggesting that Perlin noise better captures the moderate elevation shifts observed in alpine 
topography. Its negative kurtosis further mirrors that of real terrains like Everest and Matterhorn, indicating a flatter distribution with 
fewer extreme outliers. Although Value noise shows a favorable skewness profile, this similarity appears to be an artifact of randomness.

The power spectral density (PSD) analysis provides further evidence in favor of Perlin noise. Its frequency domain representation reveals 
a strong central energy concentration and a gradual decay towards higher frequencies, resembling the spectral patterns observed in terrains 
such as Everest and Denali. In contrast, Value noise produces a flatter, more uniform PSD. The absence of distinct low-frequency dominance 
and directional variation implies a lack of structural realism.

However, both synthetic models remain fundamentally limited in capturing the true complexity of real-world landscapes. Not only did both 
fail to capture the full statistical range observed in natural terrain, most notably the variance metric, its PSD stray far from resembling 
the frequency decay observed in natural terrain. 

This inquiry has significantly deepened my understanding of noise functions, particularly how their underlying mathematical foundations 
and algorithms influence their capacity to replicate the complexity and variability found in natural landscapes. Additionally, the research 
process broadened my knowledge with statistical tools and introduced me to spectral analysis and signal processing techniques, most notably 
the use of Power Spectral Density, which I had not encountered before.

Given that this area of study remains relatively underexplored, with limited literature due to the high competition between game companies, 
I often found myself navigating through a labyrinth with few established guidelines. While this was challenging, the lack of precedent also 
gave me the freedom to formulate and test original approaches. I am particularly proud of the fact that many of the techniques and comparisons 
I employed were conceptualized on my own. This experience serves more than just a stepping-stone for my problem-solving and analytical abilities 
but also affirmed my enthusiasm for mathematical research. 

Looking ahead, I am eager to further explore this topic at the university level, where I can apply more advanced and rigorous mathematical tools 
and contribute to a promising field with fresh insights and greater technical depth.  
