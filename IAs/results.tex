\section{\textsc{Results}}
\hrule height 0.5pt
\vspace*{2.5pt}

\subsection{\textsc{Calculation}}
\vspace*{-10pt}

Before using technology to produce a complete terrain map, I will show the hand calculation for one pixel $(u,v)=(3,143)$ in both Value 
and Perlin noise. The parameters will be $\lambda_0=2$, $A_0=1$, $L=2$, $P=0.5$, and $O=6$. The seed, as mentioned in the introduction, 
is $30042603$. 

Firstly, transform pixel coordinates into the domain of the noise function, then scale by $\lambda_0$ which gives:
\[\left(\frac{u}{N}\lambda_0,\frac{v}{N}\lambda_0\right)=\left(\frac{3}{256}\cdot2,\frac{143}{256}\cdot2\right)=\left(\frac{3}{128},\frac{143}{128}\right)\]
Next, find the associated top-left vertex:
\[(i,j)=\left(\left\lfloor\frac{3}{128}\right\rfloor,\left\lfloor\frac{143}{128}\right\rfloor\right)=(0,1)\]
which means the other three vertices are $(1,1)$, $(0,2)$, and $(1,2)$.

Generating random scalar values for the lattice function $G_V$ with the seed, we obtain:
\begin{equation*}
    \begin{bmatrix}
        G_V(0,1) & G_V(1,1)\\
        G_V(0,2) & G_V(1,2)
    \end{bmatrix}
    =
    \begin{bmatrix}
        0. & 0.\\
        0. & 0.
    \end{bmatrix}
\end{equation*}
Substituting these values into our bilinear interpolation (Equation \ref{eq:1}), with the cubic smoothstep function $S_3(t)$ (Equation \ref{eq:3}):
\begin{align*}
    \EuScript{N}_V\left(\frac{3}{128},\frac{143}{128}\right)&=
    \begin{bmatrix}
        1-S_n(\frac{3}{128}-0) & S_n(\frac{3}{128}-0)
    \end{bmatrix}
    \begin{bmatrix}
        G_V(0,1) & G_V(1,1)\\
        G_V(0,2) & G_V(1,2)
    \end{bmatrix}
    \begin{bmatrix}
        1-S_n(\frac{143}{128}-1)\\
        S_n(\frac{143}{128}-1)
    \end{bmatrix}\\
    &\approx a
\end{align*}
Finally, multiply the value by the amplitude to get the final elevation value at the pixel $(3,143)$.
\[A_0\cdot\EuScript{N}_V\left(\frac{3}{128},\frac{143}{128}\right)\]
Repeat this process and plot them to all the appropriate pixels using technology, we will obtain the full map for one set of amplitude and frequency. 
\subsection{\textsc{Visual Comparison}}
\vspace*{-10pt}

\subsection{\textsc{Quantitative Comparison}}
\vspace*{-10pt}