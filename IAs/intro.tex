\section{\textsc{Introduction}}
\hrule height 0.5pt
\vspace*{2.5pt}
Mathematics does not merely describe nature - it architects worlds. Within the
intricate algorithm, virtual worlds take shape as jagged mountains rise, valleys
unfurl, and rivers etch serpentine paths. This is not magic; it is mathematics in
motion, bending stochasticity into artistry. 

Traditionally, the primary technique used in developing terrain has been handcraft,
using modelling software to precisely describe the creations. However, this approach
suffers from four main drawbacks (Roden \& Parberry, 2004):
\begin{itemize}
    \item \textbf{Operability:} handcrafted terrains are slow and resource-intensive, requiring
    manual adjustments for every detail.
    \item \textbf{Inflexibility:} modifying completed designs can dramatically alter technical
    requirements.
    \item \textbf{Incompatibility:} handcrafted terrains often lack coherence with physics engines
    and other computational models.
    \item \textbf{Unscalability:} as virtual worlds grow in complexity with their expansive, dynamic
    environment, handcrafted terrains become impractical.
\end{itemize}
In the late 1960s, computer graphics pioneers confronted these setbacks, seeking ways to simulate natural
patterns without the efficiencies of manual design (Autodesk, 2024). The answer manifested in the form of
noise algorithms, mathematical functions capable of producing structured randomness, often termed ``pseudo-randomness''
(Perlin, 2001). Over the decades, they have evolved beyond gaming applications to enable hyper-realistic CGIs in films
(Pegg, 2010), creating texture (Perlin, 1985), and advancing fluid dynamics simulations (Kim et al., 2008). 

My fascination with noise algorithms began in the blocky worlds of the game Minecraft. As a teenager, I spent countless
hours exploring its vast biomes, awestruck by the seamless blend of towering cliffs, dense forests, and sprawling cave systems.
The terrain felt organic, but I knew it was generated algorithmically. While I appreciate the aesthetics of the game, I lacked
insight into the technical requirements underpinning it. Years later, I discovered the role of noise in shaping these landscapes
and became curious about which aspects of noise contribute to terrain realism. 

There are two main types of noise: Value noise and Perlin noise. Despite the popularity, their ability to replicate real-world
terrain has not been rigorously studied under controlled conditions. This IA will determine which algorithm produces more realistic
terrain by statistically comparing
\begin{itemize}
    \item Elevation distribution using \textbf{Histograms of normalised height values}.
    \item Spatial autocorrelation using \textbf{Moran's I to measure globl clustering trends}.
    \item Spectral characteristics using \textbf{Power Spectral Density to classify frequency dominance}.
\end{itemize}
with real-world elevation data from the Global Multi-resolution Terrain Elevation Data (GMTED2010), a model developed through a
collaborative effort between the U.S. Geological Survey (USGS) and the National Geospatial-Intelligence Agency (NGA). 

Terrain maps will be generated on a fixed $256\times256$ grid with normalised height values $z\in[0,1]$. Both algorithms are tested
under identical spectral parameters, including amplitude, frequency, persistence, lacunarity, and number of octaves. Cubic smoothstep
functions will be applied to bilinear interpolations to enforce continuity. A fixed seed ($30042603$) will be employed to ensure deterministic
comparisons by removing the randomness of the height values out of the effects.

Lastly, all implementations will be done using the Python programming language. All code will be available in the Appendix.