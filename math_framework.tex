\section{Mathematical Frameworks}
\hrule height 1pt
\vspace*{5pt}

\subsection{Parametric Surfaces}
\vspace*{-10pt}
A surface in three-dimensional space is often most effectively 
described using a parametric representation. In this framework, a 
surface is defined by a vector-valued function
\[
    \vec{r}(u,v)=x(u,v)\vec{i} + y(u,v)\vec{j} + z(u,v)\vec{k}
\]
where $u$ and $v$ are independent parameters that vary over a 
two-dimensional region $D$ in the $uv$-plane. The set of all such position 
vectors forms the parametric surface $S$ (Dawkins, 2024).

Unlike implicit surfaces defined by equations of the form $F(x,y,z)=0$, parametric
surfaces allow for the direct generation of points and the computation of geometric
properties such as tangent vectors through partial derivatives, which will later 
on prove itself useful in the context of cartography.

If we express the two parameters $u$ and $v$ as a function of another variable
$t$, then by applying the chain rule to $\vec{r}(u(t),v(t))$ with respect to 
$t$, we get:
\[
    \frac{d\vec{r}}{dt}=\frac{\partial \vec{r}}{\partial u}\frac{du}{dt}+\frac{\partial \vec{r}}{\partial v}\frac{dv}{dt}
\]
This derivative represents the tangent to the curve formed by $u(t)$ and $v(t)$ 
(Patrikalakis et al., 2009).

Another important property of surfaces, particularly relevant in cartography, 
is the notion of smoothness. Depending on the desired level of accuracy, the 
Earth's surface may be modeled in various ways: as a perfect sphere, an oblate 
spheroid, or a more complex geoid that represents mean sea level variations. 
Regardless of the chosen model, a key geometric feature of such surfaces is 
that, when observed at an infinitesimally small scale, they resemble a flat 
plane. In other words, the surface is locally planar, a property that 
characterizes it as smooth (Ozuch, 2024).

As mentioned in the introduction, we will model the Earth as a perfect sphere
due to the limited scope of this Extended Essay. Let $\phi$ and $\lambda$ be
the longitude and latitude of the Earth, acting as parameters of the surface, 
and let $M(\phi,\lambda)$ be a point on the Earth's surface. 

<insert diagram here>

Fix the center of the spherical Earth at the Cartesian coordinate $(0,0,0)$, with
the poles aligning with the $z$-axis. Let $M'$ be the projection of point $M$ onto
the $xy$-plane in 3D Cartesian space. We can see that the $z$-coordinate of $M$
would be the length $MM'=R\sin\lambda$.

To find the $x$ and $y$-coordinate of $M$, we can also use trigonometry and obtain
\begin{align}
    x&=OM'\cos\lambda=R\cos\lambda\cos\phi\\
    y&=OM'\sin\lambda=R\cos\lambda\sin\phi
\end{align}