\documentclass[12pt]{article}

\usepackage[english]{babel}
\usepackage{amsmath, amsfonts, mathtools, amsthm, amssymb}

% margin of paper
\usepackage[a4paper, portrait, margin=1.25in]{geometry}
\usepackage{layout}

% font: Times New Roman
\usepackage{mathptmx}

% Counter for definition
\newcounter{defcounter}

% Definition environment
\newenvironment{definition}
    {
        \\ [0.5cm]
        \begin{tabular}{|p{1\textwidth}|}
        \hline \\ [-1ex]
        \refstepcounter{defcounter}
        \textbf{Definition \thedefcounter.}
    }
    { 
        \\ [1.5ex]
        \hline
        \end{tabular}
    }
\numberwithin{defcounter}{section}

\title{ 
    \normalsize \textsc{IB Mathematics Extended Essay} \\ [2.5cm]

	\LARGE MAP PROJECTIONS
	\rule{\linewidth}{0.5pt} \\
	\Large \textbf{How do we quantify geometric distortions of different map projections using differential geometry?}
	\rule{\linewidth}{1pt} \\ [1cm]
	\normalsize \today \vspace*{5\baselineskip}
}

\date{}
\author{Gia Phu Huynh}
\begin{document}

\maketitle

\pagebreak
\tableofcontents

\pagebreak
\raggedright
\section{Introduction}
\hrule height 1pt
\vspace*{5pt}
Lorem ipsum dolor sit amet, consectetur adipiscing elit. 
Morbi fermentum aliquam nulla, at auctor diam molestie ut. 
Nullam eget orci eu mi accumsan suscipit. 
Mauris finibus mauris eu dui maximus cursus. 
Vivamus ornare cursus sem, eu auctor massa. 
In at venenatis massa. Interdum et malesuada fames ac ante 
ipsum primis in faucibus. Aliquam at tristique libero. 
Aliquam semper scelerisque justo ornare ullamcorper. 
Interdum et malesuada fames ac ante ipsum primis in faucibus. 
Donec elementum sagittis urna et maximus.

\begin{definition}{Cauchy Inequality}
	For any list of reals $u_1, u_2, \ldots, u_n$ and $v_1, v_2, \ldots, v_n$,
	\[
	\left(\sum_{i=1}^{n}u_iv_i\right)^2 \le 
	\left(\sum_{i=1}^{n}u_i^2\right)
	\left(\sum_{i=1}^{n}v_i^2\right),
	\]
	with equality if and only if there exists a constant $t$ 
	such that $u_i = t v_i$ for all $1 \leq i \leq n$, or if one list consists of only zeroes. 
\end{definition}

Integer eu leo fringilla, ultrices odio vitae, commodo nisi. 
Nunc a odio at nunc porttitor tincidunt. Etiam at sem sit 
amet lorem volutpat convallis. Fusce id malesuada tellus. 
Aenean volutpat nulla auctor faucibus mollis. Maecenas 
hendrerit justo iaculis tortor dictum vehicula at vitae 
leo. Nam ultricies tellus at eleifend egestas. Proin laoreet 
eget tortor sit amet iaculis. Praesent et enim elementum, 
egestas elit et, ultricies dui. Donec quis purus arcu. Nunc 
augue risus, consequat id augue convallis, iaculis 
facilisis diam. Pellentesque diam lacus, sagittis sed 
diam ut, efficitur blandit nulla. In non elit id dolor 
porta pulvinar. Nam a sollicitudin velit, quis dapibus nisl.

Nullam pretium, elit eget bibendum efficitur, massa nisi accumsan 
sem, vel rhoncus tellus dui sed est. Etiam finibus, justo tristique 
mattis molestie, ex enim laoreet nulla, id mattis metus urna nec 
ante. Integer tempus felis id odio condimentum, eget finibus sem 
tempor. Donec commodo cursus iaculis. In at sem non velit suscipit 
volutpat eget id mauris. Ut volutpat, eros varius lacinia congue, 
erat elit iaculis nunc, in laoreet nibh risus viverra sem. Duis 
venenatis dictum suscipit.

Sed pulvinar ligula velit, eu interdum eros aliquam eget. 
Curabitur semper interdum diam at elementum. Suspendisse in 
dignissim libero, ac imperdiet dolor. Donec nibh mauris, eleifend 
a lectus id, euismod commodo eros. Quisque iaculis leo est, eget 
commodo metus consectetur vestibulum. Pellentesque elementum 
tristique libero. Mauris sollicitudin quam vitae magna venenatis 
tempor. Duis euismod turpis quis ex maximus porta. Maecenas sapien 
elit, sollicitudin nec velit non, sollicitudin viverra nulla.

\begin{example}{Graphing a hyperbola}
	{
		Sketch the hyperbola given by 
		$\dfrac{(y-2)^2}{25}-\dfrac{(x-1)^2}{4}=1$.
	}
	The hyperbola is centered at $(1, 2)$; $a=5$ and $b=2$. We draw 
	the prescribed rectangle centered at $(1, 2)$ along with the 
	asymptotes defined by its diagonals. The hyperbola has a vertical 
	transverse axis, so the vertices are located at $(1, 7)$ and 
	$(1, -3)$. 

	\hspace*{20pt}We also find the location of the foci: as $c^2 = a^2 + b^2$, we 
	have $c = \sqrt{29} \approx 5.4$. Thus the foci are located at $(1, 2 \pm 5.4)$ 
	as shown in the figure below.
	\begin{center}
				
	\end{center}
\end{example}
\end{document}